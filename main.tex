\documentclass{article}
\usepackage{graphicx} % Required for inserting images
\usepackage{amsmath}
\usepackage{amssymb}

\title{Laplace Transform, something that always bothered me and i was not quite sure about it.}
\author{Tin Novosel}
\date{September 2024}

\begin{document}
\maketitle

When I was younger, I played a lot of video games. Cheat codes were always there to make the experience much more enjoyable. I had the same feeling when I first started using Laplace transforms to solve differential equations—it felt like another life hack. For a long time, however, I wasn’t sure where the whole idea came from.
\vspace{1cm}

\noindent One way to demystify the topic is to start with \textbf{power series}.

\begin{equation}
\label{eq:powerserie}
    \sum_{n=0}^{\infty} a_n x^n = a_0 + a_1 x + a_2 x^2 + \cdots
\end{equation}\\
 \noindent To later the whole story becoming much more familiar, we will give generic name to the series as $A(x)$.\\

 \begin{equation}
 \label{eq:markedps}
     \sum_{n=0}^{\infty} a_n x^n = A(x)
 \end{equation}\\

 \noindent As the $n \in \mathbb{N^+}$ and $a \in \mathbb{R}$, $n$ indexed parameter can be viewed as the discrete function $a(n)$ and equation (\ref{eq:powerserie}) becomes:\\
 
 \begin{equation}
 \label{eq:markedpss}
     \sum_{n=0}^{\infty} a(n) x^n = A(x)
 \end{equation}\\

\newpage

\noindent Now, we can think of equation (\ref{eq:markedpss}) as associating the discrete function of the parameters of the power series with the sum of the power series. This can also be viewed as taking a discrete function defined for positive integers and converting it into a continuous function.\\

\begin{equation}
    a(n) \rightarrow A(x)
\end{equation}
\\
Some examples of this could be, if we take as discrete function:
\begin{itemize}
    \item $a(n) = 1$, left hand sight then becomes $1 + x + x^2 + x^3 + ...$ and from \textit{Maclaurin Expansions} we can transform power series to this exprassion $\frac{1}{1-x}$ with restricted domain to $|x| < 1$.

    \item $a(n) = \frac{1}{n!}$, power series is then $1 + \frac{x}{1!} + \frac{x^2}{2!} + \frac{x^3}{3!} + ...$  and this expansion equals to well known $e^x$\\
    \\
\end{itemize}

Now, even though it's not motivated, let's do the continuous analog of the power series, where we are allowed to take every real value, not just positive integers.\\
\begin{equation}
    n = 0,1,2,3,... \rightarrow t \in {0 \leq t < \infty}
\end{equation}\\

\noindent Summation then becomes integral over real numbers and $n$ in replaced with continuous variable $t$.\\ 

\begin{equation}
\label{eq:conti}
    \int_0^{\infty} a(t)x^t \, dt = A(x)
\end{equation}\\

\noindent Having $x$ as a base of the exponential is not very practical  when solving the integration, so we will modify equation (\ref{eq:conti}) to make it more convenient by using the natural base $e$.

\begin{equation}
    x^t = e^{(ln(x))t}
\end{equation}\\

\noindent Also, we want to calculate this improper integral and in order to make it converging we have to restrict the $x$.\\
\begin{equation}
\label{eq:xrange}
    0 < x < 1 \rightarrow ln(x) < 0
\end{equation}

\newpage

The variable $ln(x)$ is not very aesthetically pleasing to have in equation so the infamous \textbf{s} comes into play! This substitution looks like:\\
\begin{equation}
    -s = ln(x)
\end{equation}\\

\noindent The minus sign on the left side ensures that $s$ is always positive due to (\ref{eq:xrange}). Additionally, functions are usually denoted as $f(t)$ rather than $a(t)$, so we will make that change as well, purely for the looks. After all these cosmetic changes, we finally end up with:\\

\begin{equation}
\label{eq:laplace}
    \int_0^{\infty} f(t)e^{-st} \, dt = F(s)
\end{equation}\\

\noindent This expression (\ref{eq:laplace}), called \textbf{Laplace transform} is nothing more than a \textbf{continuous analog to the summation of the power series}. Which takes function defined for the positive real values of \textbf{t} and converts it into the function of variable \textbf{s}. 

\vfill 
\begin{center}
\textit{Derivation of Laplace transform is inspired by Prof. Arthur Mattuck,\\ 
of the Department of Mathematics, MIT lecture.}
\end{center}

 
\end{document}
